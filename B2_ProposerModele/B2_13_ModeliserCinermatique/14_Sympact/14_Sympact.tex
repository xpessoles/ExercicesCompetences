\normalfalse \difficiletrue \tdifficilefalse
\correctionfalse

%\UPSTIidClasse{11} % 11 sup, 12 spé
%\newcommand{\UPSTIidClasse}{12}

\exer{Barrière Sympact $\star\star$ \label{B2:13:14}}
\setcounter{numques}{0}
\UPSTIcompetence{B2-13}
\index{Compétence B2-13}
\index{Barrière Sympact}
\ifcorrection
\else
\textbf{Pas de corrigé pour cet exercice.}
\fi

\ifprof
\else
Soit le mécanisme suivant. On a $\vect{AC}=H\vect{j_0}$ et $\vect{CB}=R\vect{i_1}$. De plus, 
$H=\SI{120}{mm}$ et $R=\SI{40}{mm}$. 

\begin{center}
\includegraphics[width=\linewidth]{14_01}
\end{center}
\fi


\question{Calculer $\vectv{B}{1}{0}$ ?}
\ifprof
\else
\fi

\question{Calculer $\vectv{B}{2}{0}$ ?}
\ifprof
\else
\fi

\question{Justifier que $\vectv{B}{2}{1}\cdot\vect{j_2}=\vect{0}$.}
\ifprof
\else
\fi

\question{En déduire une relation cinématique entre les différentes grandeurs.}
\ifprof
\else
\fi


%\question{En déduire la course de la pièce \textbf{3}.}
%\ifprof
%\else
%\fi



\ifprof
\else
\begin{flushright}
\footnotesize{Corrigé  voir \ref{B2:13:14}.}
\end{flushright}%
\fi