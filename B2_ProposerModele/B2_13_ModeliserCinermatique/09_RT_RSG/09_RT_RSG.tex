\normalfalse \difficiletrue \tdifficilefalse
\correctionfalse

%\UPSTIidClasse{11} % 11 sup, 12 spé
%\newcommand{\UPSTIidClasse}{12}

\exer{Mouvement RT -- RSG  $\star\star$ \label{B2:13:09}}
\setcounter{numques}{0}
\UPSTIcompetence[2]{B2-13}
\index{Compétence B2-13}
\index{Mécanisme à 1 rotations, 1 translation et RSG}
\ifcorrection
\else
\textbf{Pas de corrigé pour cet exercice.}
\fi

\ifprof
\else
Soit le mécanisme suivant. On a $\vect{IA}=R\vect{j_0}$ et $\vect{AB}=\lambda(t)\vect{i_1}$. De plus $R=\SI{15}{mm}$.
On fait l'hypothèse de roulement sans glissement au point $I$.
\begin{center}
\includegraphics[width=\linewidth]{09_RT_RSG_01}
\end{center}
\fi


\question{Déterminer $\vectv{B}{2}{0}$.}
\ifprof
\else
\fi

\question{Donner le torseur cinématique $\torseurcin{V}{2}{0}$ au point $B$.}
\ifprof
\else
\fi

\question{Déterminer $\vectg{B}{2}{0}$.}
\ifprof
\else
\fi



\ifprof
\else
\begin{flushright}
\footnotesize{Corrigé  voir \ref{B2:12:13}.}
\end{flushright}%
\fi