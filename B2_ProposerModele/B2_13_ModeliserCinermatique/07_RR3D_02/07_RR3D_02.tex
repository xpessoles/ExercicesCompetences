\normaltrue \difficilefalse \tdifficilefalse
\correctionfalse

%\UPSTIidClasse{11} % 11 sup, 12 spé
%\newcommand{\UPSTIidClasse}{12}

\exer{Mouvement RR 3D  $\star$ \label{B2:13:07:02}}
\setcounter{numques}{0}
\UPSTIcompetence{B2-13}
\index{Compétence B2-13}
\index{Mécanisme à 2 rotations 3D}
\ifcorrection
\else
\textbf{Pas de corrigé pour cet exercice.}
\fi

\ifprof
\else
Soit le mécanisme suivant. On a $\vect{AB}=R\vect{i_1}$ et $\vect{BC}=\ell\vect{i_2}+r\vect{j_2}$. On note $R+\ell=L = \SI{20}{mm}$ et $r=\SI{10}{mm}$.
\begin{center}
\includegraphics[width=\linewidth]{07_RR3D_01}
\end{center}
\fi

\question{Déterminer $\vectv{C}{2}{0}$ par dérivation vectorielle.}
\ifprof ~\\
$\vectv{C}{2}{0}$ 
$=\deriv{\vect{AC}}{\rep{0}}$ 
$=\deriv{R\vi{1}+\ell\vi{2}+r\vj{2}}{\rep{0}}$.

Calculons : 
\begin{itemize}
\item $\deriv{\vi{1}}{\rep{0}}$ $=\vecto{1}{0} \wedge \vi{1}$ $=\dot{\theta}\vk{0} \wedge \vi{1}$ $=\dot{\theta}\vj{1}$.
\item $\deriv{\vi{2}}{\rep{0}}$ $=\dot{\theta}\vj{1}$ ($\vi{1}=\vi{2}$).
\item $\deriv{\vj{2}}{\rep{0}}$ $=\vecto{2}{0} \wedge \vj{2}$ 
$=\left(\dot{\theta}\vk{0}+ \dot{\varphi}\vi{1} \right) \wedge \vj{2}$
$=\dot{\theta}\vk{1} \wedge \vj{2}+ \dot{\varphi}\vi{1}  \wedge \vj{2}$
$=-\dot{\theta}\sin\varphi \vi{1} + \dot{\varphi}\vk{2} $.
\end{itemize}

On a donc, 
$\vectv{C}{2}{0}= \left(R+\ell\right)\dot{\theta}\vj{1} -r\dot{\theta}\sin\varphi \vi{1} + r\dot{\varphi}\vk{2}$.
\else
\fi


\question{Déterminer $\vectv{C}{2}{0}$ par composition.}
\ifprof ~\\
On a $\vectv{C}{2}{0}= \vectv{C}{2}{1}+\vectv{C}{1}{0}$.
\begin{itemize}
\item $\vectv{C}{2}{1}$ : on passe par $B$ car $B$ est le centre de la pivot entre 2 et 1 et que $\vectv{B}{2}{1}=\vect{0}$. 
$\babarv{C}{B}{2}{1} =\left(-\ell\vi{2}-r\vj{2} \right)\wedge \dot{\varphi}\vi{1} $ 

$=-\ell\vi{2}\wedge \dot{\varphi}\vi{1}-r\vj{2} \wedge \dot{\varphi}\vi{1}$.

$=r\dot{\varphi}\vk{2}$.

\item $\vectv{C}{1}{0}$ : on passe par $A$ car $A$ est le centre de la pivot entre 1 et 0 et que $\vectv{A}{1}{0}=\vect{0}$ est nul. 
$\babarv{C}{A}{1}{0}$ 
$=\left(-r\vj{2}-\ell\vi{2}-R\vi{1}\right)\wedge \left(\dot{\varphi}\vk{1}+\dot{\theta}\vj{1}\right)$

%$=\left(-r\vj{2}-\ell\vi{2}-R\vi{1}\right)\wedge\dot{\varphi}\vk{1}
%+\left(-r\vj{2}-\ell\vi{2}-R\vi{1}\right)\wedge \dot{\theta}\vj{1}$%%

%$=\left(-r\vj{2}\wedge\dot{\varphi}\vk{1}-\ell\vi{2}\wedge\dot{\varphi}\vk{1}-R\vi{1}\wedge\dot{\varphi}\vk{1}\right)
%+\left(-r\vj{2}\wedge \dot{\theta}\vj{1}-\ell\vi{2}\wedge \dot{\theta}\vj{1}-R\vi{1}\wedge \dot{\theta}\vj{1}\right)$%%

$=-r\dot{\varphi}  \sin\varphi \vi{1} +\ell\dot{\varphi}\vj{1}+R\dot{\varphi}\vj{1}
+r\sin\varphi\dot{\theta}\vi{1}-\ell \dot{\theta}\vk{1}-R \dot{\theta}\vk{1}$%%
\end{itemize}

Au final, 
 $\vectv{C}{2}{0}=r\dot{\varphi}\vk{2}-r\dot{\varphi}  \sin\varphi \vi{1} +\ell\dot{\varphi}\vj{1}+R\dot{\varphi}\vj{1}
+r\sin\varphi\dot{\theta}\vi{1}-\ell \dot{\theta}\vk{1}-R \dot{\theta}\vk{1}$.
\else
\fi

$\vectv{C}{2}{0}= \left(R+\ell\right)\dot{\theta}\vj{1} -r\dot{\theta}\sin\varphi \vi{1} + r\dot{\varphi}\vk{2}$.

\question{Donner le torseur cinématique $\torseurcin{V}{2}{0}$ au point $C$.}
\ifprof ~\\
\else
\fi

\question{Déterminer $\vectg{C}{2}{0}$.}
\ifprof ~\\
\else
\fi


\ifprof
\else
\begin{flushright}
\footnotesize{Corrigé  voir \ref{B2:13:07}.}
\end{flushright}%
\fi