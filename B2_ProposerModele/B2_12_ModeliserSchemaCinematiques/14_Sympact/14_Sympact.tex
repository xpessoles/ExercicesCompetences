\normalfalse \difficiletrue \tdifficilefalse
\correctionfalse

%\UPSTIidClasse{11} % 11 sup, 12 spé
%\newcommand{\UPSTIidClasse}{12}

\exer{Barrière Sympact $\star\star$ \label{B2:12:14}}
\setcounter{numques}{0}
\UPSTIcompetence{B2-12}
\index{Compétence B2-12}
\index{Barrière Sympact}
\ifcorrection
\else
\textbf{Pas de corrigé pour cet exercice.}
\fi

\ifprof
\else
Soit le mécanisme suivant. On a $\vect{AC}=H\vect{j_0}$ et $\vect{CB}=R\vect{i_1}$. De plus, 
$H=\SI{120}{mm}$ et $R=\SI{40}{mm}$. 

\begin{center}
\includegraphics[width=\linewidth]{14_01}
\end{center}
\fi


\question{Tracer le graphe des liaisons.}
\ifprof
\begin{center}
\includegraphics[width=\linewidth]{14_02_c}
\end{center}
\else
\fi

\question{Retracer le schéma cinématique pour $\theta(t)=\dfrac{\pi}{2}\,\text{rad}$.}
\ifprof
\begin{center}
\includegraphics[width=\linewidth]{14_01_c}
\end{center}

\else
\fi

\question{Retracer le schéma cinématique pour $\theta(t)=75\degres$.}
\ifprof
\else
\fi


\question{Dans l'hypothèse où la pièce \textbf{1} peut faire des tours complets, quelle doit être la longueur minimale de la pièce \textbf{2}.}
\ifprof
\else
\fi

\question{Dans l'hypothèse où la pièce \textbf{2} fait \SI{12}{cm}, quel sera le débattement maximal de la pièce \textbf{1}.}
\ifprof
\else
\fi



\ifprof
\else
\begin{flushright}
\footnotesize{Corrigé  voir \ref{B2:12:14}.}
\end{flushright}%
\fi