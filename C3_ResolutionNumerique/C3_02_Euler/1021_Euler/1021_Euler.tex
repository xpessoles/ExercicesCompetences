\normaltrue \difficilefalse \tdifficilefalse
\correctionfalse

%\UPSTIidClasse{11} % 11 sup, 12 spé
%\newcommand{\UPSTIidClasse}{11}

\exer{Schéma d'Euler$\star$ \label{C3:02:1021}}
\setcounter{numques}{0}
\UPSTIcompetence[2]{C3-02}
\index{Compétence C3-02}
\index{Schéma d'Euler explicite}
\ifcorrection
\else
\textbf{Pas de corrigé pour cet exercice.}
\fi


\question{Donner la méthode de résolution numérique des équations différentielles suivantes en utilisant le schéma d'Euler explicite.}
\begin{eqnarray}
\left\{
\begin{array}{l}
y'(t) + \alpha y(t) = \beta \\
y(0) = \gamma
\end{array}
\right.
\end{eqnarray}


\ifprof

\subsection*{Équation 1}
On a :
$$
y'(t) \simeq \dfrac{y(t+h)-y(h)}{h}
$$
En discrétisant le problème, on a $y_k=y(kh)=y(t)$; donc : 
$$
\dfrac{y(t+h)-y(h)}{h} + \alpha y(t) = \beta \Longrightarrow 
\dfrac{y_{k+1}-y_k}{h} + \alpha y_k = \beta \Longleftrightarrow 
y_{k+1}  = \beta h - \alpha y_k + y_k \Longleftrightarrow 
y_{k+1}  = \beta h + y_k\left(1 - \alpha \right) 
$$
\else
\fi



 

\ifprof
\else
\begin{flushright}
\footnotesize{Corrigé  voir \ref{C3:02:1021}.}
\end{flushright}%
\fi