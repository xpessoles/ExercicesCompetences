\normaltrue
\correctionfalse

%\UPSTIidClasse{11} % 11 sup, 12 spé
%\newcommand{\UPSTIidClasse}{11}


\exer{Mouvement TT -- $\star$ \label{C2:09:03}}
\setcounter{numques}{0}
\UPSTIcompetence[2]{C2-09}
\index{Compétence C2-09}
\index{Principe fondamental de la dynamique}
\index{PFD}
\index{Mécanisme à 2 translations}
\ifcorrection
\else
\textbf{Pas de corrigé pour cet exercice.}
\fi

\ifprof
\else
Soit le mécanisme suivant. On note $\vect{AB}=\lambda(t)\vect{i_0}$ et $\vect{BC}=\mu(t)\vect{j_0}$.
$G_1 = B$ désigne le centre d'inertie de \textbf{1},et $m_1$ sa masse et $\inertie{G_1}{1}=\matinertie{A_1}{B_1}{C_1}{0}{0}{0}{\bas{1}}$; 
$G_2 = C$ désigne le centre d'inertie de \textbf{2} et  $m_2$ sa masse  et $\inertie{G_2}{2}=\matinertie{A_2}{B_2}{C_2}{0}{0}{0}{\bas{2}}$.

 Un vérin électrique positionné entre \textbf{0} et \textbf{1} permet d'actionner le solide \textbf{1}.
 Un vérin électrique positionné entre \textbf{1} et \textbf{2} permet d'actionner le solide \textbf{2}.

L'accélération de la pesanteur est donnée par $\vect{g}=-g\vect{j_0}$.

\begin{center}
\includegraphics[width=.6\linewidth]{03_TT_01}
\end{center}
\fi

\question{Dans le but d'obtenir les lois de mouvement, appliquer le théorème de la résultante dynamique au solide \textbf{2} en projection sur $\vect{j_0}$ puis  le théorème de la résultante dynamique à l'ensemble \textbf{1+2} en projection sur $\vect{i_0}$}
\ifprof
\else
\fi


\ifprof
\else
\begin{flushright}
\footnotesize{Corrigé  voir \ref{C2:09:03}.}
\end{flushright}%
\fi