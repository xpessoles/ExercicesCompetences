\normaltrue \difficilefalse \tdifficilefalse
\correctionfalse
%\UPSTIidClasse{11} % 11 sup, 12 spé
%\newcommand{\UPSTIidClasse}{11}

\exer{Banc hydraulique $\star$ \label{C2:03:prec:63}}
%% CCP MP 2010
\setcounter{numques}{0}
\UPSTIcompetence[2]{C2-03}
\index{Compétence C2-03}
\index{Schéma-blocs}
\index{Précision}

\ifcorrection
\else
\textbf{Pas de corrigé pour cet exercice.}
\fi

\ifprof
\else

Pour limiter l’erreur statique due aux fuites, on envisage d’asservir la pression d’eau dans le tube. 
%L’objectif est ici de proposer un réglage du correcteur pour répondre aux critères du cahier des charges.
La pression d’eau à l’intérieur du tube est mesurée par un capteur de pression. 

\begin{center}
\includegraphics[width=\linewidth]{63_01}
\end{center}

 
 \begin{tabular}{lp{5cm}}
$P_{\text{con}}(p)$ : & 	pression de consigne d’eau dans le tube (Pa) \\
$P_e(p)$ : & 	pression d’eau dans le tube (Pa) \\
$U_c(p)$ : & 	tension de commande du régulateur de pression (V)\\
$P_r(p)$ : &	pression d’huile régulée (Pa)\\
$\Delta Q_e(p)$ :& 	débit de fuite (\si{m^3s^{-1}})\\
$U_m(p)$ 	:&	tension de mesure du capteur (V)\\
\end{tabular}
 
 Hypothèses :
\begin{itemize}
\item Ll’ensemble de mise sous pression {tube + distributeur + multiplicateur de pression} est défini par les transmittances suivantes : $H_{\text{pre}} (p)=\dfrac{K_m}{1+T_1 p}$	et	$H_{\text{fui}} (p)=\dfrac{K_f}{1+T_1 p}$ avec 	$K_m = 3,24$ ; 	$K_f = \SI{2,55e10}{Pa.m^{-3}.s}$ ; 	$T_1  = \SI{10}{s}$.
\item L’ensemble {pompe+régulateur de pression} est modélisé par la fonction de transfert :
$H_{\text{pom}} (p)=\dfrac{K_{\text{pom}}}{1+T_2 p}$  avec 	$K_{\text{pom}} = \SI{1,234e7}{Pa/V}$; 	$T_2 = \SI{5}{s}$.
\item Le capteur est modélisé par un gain pur :	$K_{\text{cap}} = \SI{2,5e-8}{V/Pa}$.
\end{itemize}
La pression de consigne est de $P_{\text{con}} = \SI{800}{bars}$ et les débits de fuite sont estimés à $\Delta Q_e = \SI{5e-4}{m3/s}$.

 
Le cahier des charges concernant le réglage de la pression de test est le suivant.
\begin{center}
\begin{tabular}{lp{5cm}}
\hline 
Stabilité :  & marge de phase de 60\degres  \\
  	  &  marge de gain de \SI{12}{dB} \\ \hline
Rapidité :  &  temps d’établissement te < 40 s \\ \hline
Précision : & 	erreur statique < 5\% soit pour une consigne de 800 bars : \\
&erreur statique due à la consigne : $\varepsilon_{\text{con}}< 5\%$  \\
& erreur statique due à la perturbation $\varepsilon_{\text{pert}} < \SI{40}{bars}$ \\ \hline
Amortissement :&	pas de dépassement \\ \hline
\end{tabular}
\end{center}

Dans le cas d’un système bouclé convenablement amorti, on pourra utiliser, sans aucune justification, la relation :
$t_e \cdot \omega_{\SI{0}{dB}}=3$ où $\omega_{\SI{0}{dB}}$ désigne la pulsation de coupure à \SI{0}{dB} en boucle ouverte et $t_e$ le temps d’établissement en boucle fermée vis-à-vis d’un échelon de consigne :
\begin{itemize}
\item $t_e = t_m$, temps du 1er maximum si le dépassement est supérieur à \SI{5}{\%},
\item $t_e = t_R$, temps de réponse à \SI{5}{\%} si le dépassement est nul ou inférieur à \SI{5}{\%}.
\end{itemize}

On se propose de corriger le système avec le correcteur défini sur le schéma bloc ci-dessous.

\begin{center}
\includegraphics[width=5cm]{63_02}
\end{center}
\fi

\question{Déterminer la fonction de transfert $C(p)$ de ce correcteur.}
\ifprof
\else 
\fi


\question{Tracer l’allure de son diagramme de Bode en fonction des coefficients $K_i$ et $K_p$.}
\ifprof
\else 
\fi


\question{Quelle est l’influence d’un tel correcteur sur la précision et la stabilité ? Justifier.}
\ifprof
\else 
\fi


\question{Quelle valeur faut-il donner à $\omega_{\SI{0}{dB}}$ pour répondre au critère de rapidité du cahier des charges ?}
\ifprof
\else 
\fi


\question{Déterminer alors le rapport $T=\dfrac{K_p}{K_i}$ pour obtenir la marge de phase spécifiée dans le cahier des charges.}
\ifprof
\else 
\fi


\question{En déduire les valeurs de $K_i$ et $K_p$ qui permettent de régler rapidité et marge de phase.}
\ifprof
\else 
\fi

On donne les diagrammes de Bode en gain et en phase de la fonction de transfert en boucle ouverte corrigée avec le correcteur Proportionnel Intégral déterminé précédemment. On donne sa réponse temporelle avec et sans débit de fuite pour une pression de consigne d’eau de 800 bars.

\question{La réponse du système est-elle satisfaisante au regard du cahier des charges ? Justifier.}
\ifprof
\else 
\fi

\ifprof
\else

\begin{center}
\includegraphics[width=\linewidth]{63_03}
\end{center}


\begin{center}
\includegraphics[width=\linewidth]{63_04}
\end{center}
\fi

\ifprof
\else

%\noindent\footnotesize
%\fbox{\parbox{.9\linewidth}{
%Éléments de corrigé : 
%\begin{enumerate}
%  \item $\varepsilon_{\text{con \%}} = \dfrac{1}{1+K_PK_m K_{\text{pom}} K_{\text{cap}} }$;
%  \item $K_P > 19$;
%  \item $\varepsilon_{\text{pert}} = \Delta Q_e \dfrac{K_f}{1+K_{\text{cap}}K_PK_mK_{\text{pom}}}$;
%  \item $K_P > 2,19$.
%  \item $K_P < 0,125$. Il est impossible de vérifier les trois conditions avec un correcteur proportionnel.
%\end{enumerate}}}
%\normalsize

\begin{flushright}
\footnotesize{Corrigé  voir \ref{C2:03:prec:63}.}
\end{flushright}%
\fi