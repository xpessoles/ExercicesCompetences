\normalfalse \difficiletrue \tdifficilefalse
\correctionfalse

%\UPSTIidClasse{11} % 11 sup, 12 spé
%\newcommand{\UPSTIidClasse}{11}

\exer{Suspension automobile  $\star\star$ \label{C2:07:55}}
\setcounter{numques}{0}
\UPSTIcompetence[2]{C2-07}
\index{Compétence C2-07}
\index{PFS}
\index{Suspension automobile}
\ifcorrection
\else
\textbf{Pas de corrigé pour cet exercice.}
\fi

\ifprof
\else
On s'intéresse à la liaison entre l'axe de la toue et le châssis du véhicule. Les notations adoptées seront les suivantes : $F^a_{C}$ (respectivement $F^r_{C}$, $F^x_{C}$) désignera la composante suivant $\vect{a}$ (respectivement $\vect{r}$, $\vect{x}$) de l'effort extérieur exercé en $C$. On procédera de même pour le point $D$. 
\begin{center}
\includegraphics[width=\linewidth]{55_01}
\end{center}


\fi

\question{En isolant l'ensemble \{pneumatique + jante + axe de roue\}, écrire les équations issues du principe fondamental de la statique appliqué au point $C$, en projection sur les axes de la base $\base{a}{r}{x}$ en fonction des composantes $F_{\text{sol}}^a$ et $F_{\text{sol}}^r$ et des dimensions $d_0$, $d_3$ et $d_4$.}
\ifprof
\else
\fi



\question{Résoudre littéralement le système.}
\ifprof
\else
\fi
\ifprof
\else
\begin{flushright}
\footnotesize{Corrigé  voir \ref{C2:07:55}.}
\end{flushright}%
\fi 